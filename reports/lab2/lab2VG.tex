%%%%%%%%%%%%%%%%%%%%%%%%%%%%%%%%%%%%%%%%%
% fphw Assignment
% LaTeX Template
% Version 1.0 (27/04/2019)
%
% This template originates from:
% https://www.LaTeXTemplates.com
%
% Authors:
% Class by Felipe Portales-Oliva (f.portales.oliva@gmail.com) with template 
% content and modifications by Vel (vel@LaTeXTemplates.com)
%
% Template (this file) License:
% CC BY-NC-SA 3.0 (http://creativecommons.org/licenses/by-nc-sa/3.0/)
%
%%%%%%%%%%%%%%%%%%%%%%%%%%%%%%%%%%%%%%%%%

%----------------------------------------------------------------------------------------
%	PACKAGES AND OTHER DOCUMENT CONFIGURATIONS
%----------------------------------------------------------------------------------------

\documentclass[
	11pt, % Default font size, values between 10pt-12pt are allowed
	%letterpaper, % Uncomment for US letter paper size
	%spanish, % Uncomment for Spanish
]{fphw}

% Template-specific packages
\usepackage[utf8]{inputenc} % Required for inputting international characters
\usepackage[T1]{fontenc} % Output font encoding for international characters
\usepackage{mathpazo} % Use the Palatino font

\usepackage{graphicx} % Required for including images

\usepackage{booktabs} % Required for better horizontal rules in tables

\usepackage{listings} % Required for insertion of code

\usepackage{enumerate} % To modify the enumerate environment


\lstset{
	frame=single,
	breaklines=true,
	basicstyle=\footnotesize
	}


%----------------------------------------------------------------------------------------
%	ASSIGNMENT INFORMATION
%----------------------------------------------------------------------------------------

\title{Assignment 2 - VG part} % Assignment title

\date{February 17th, 2022} % Due date

\author{Dominik Künkele}

\institute{University of Gothenburg} % Institute or school name

\class{Dialogue Systems (LT2216)} % Course or class name

%----------------------------------------------------------------------------------------

\begin{document}

\maketitle % Output the assignment title, created automatically using the information in the custom commands above

%----------------------------------------------------------------------------------------
%	ASSIGNMENT CONTENT
%----------------------------------------------------------------------------------------

\section*{Improvements}

\subsection*{duplicate state nodes}
All state nodes which ask for an answer have the same structure. There is a \emph{prompt} state, in which
the machine asks the question. After the machine is finished uttering, it moves on to the \emph{listen} state
in which it listens to a response of the user. When the machine recognised something, it either moves
to another state or utters a clarification request if something wasn't understood. If the user didn't 
answer at all, it repeats the question. The only difference over all prompts is, how the machine 
handles the answer and to which state the \emph{FSM} moves on. With a naive aproach, this results 
in a lot of duplicate code for each state node for prompts. If one would like to add for example 
a \emph{pause} state in the prompt machine, this would need to be added in all state nodes.

To tackle this problem, I tried to encapsulate the prompt machine in a function 
\emph{abstractPromptMachine}. This holds all the basic necessary states and can be extended 
with transitions or additional information. There is for example the \emph{binaryPromptMachine}, 
which uses two conditions to decide, in which state the \emph{FSM} should move on. Furthermore, 
I implemented the \emph{categoryPromptMachine}, which can handle different assignments of context 
variables based on multiple grammars. Lastly, the \emph{abstractPromptMachine} can also be extended
directly in the state node defintion, as done for example for the main menu.

This approach gives the developer a lot more flexibility, to do changes necessary for all or 
only some specific prompt machines.


\subsection*{non-flexible grammar}
Defining a grammar, which translates each possible utterance character by character into the target
text has many disadvantages: First of all, the voice recognition can change. When in one NLU module
an uttered sentence may be recognized as "Thank you.", in another NLU module it may be recognized 
as "Thank you!". A change of one character may invalidate the current implemented grammar. 
This could be handled, by extending the grammar with many possible different ways how the sentences could be
uttered. That would lead to the next problem: There are far too many possible ways, to include each
in the grammar. This becomes especially visible, when trying to parse an uttered time. Just including 
all different minutes of an hour would be a huge effort and would bloat the grammar.

A solution for this could be, to define the grammar in a more flexible way. I tried several different
approaches. For the grammar for \emph{yes/no-questions}, I used the basic approach of just writing out the
different possible ways of accepting or declining, since the user will most likely only use a few 
different. But still, the machine can't handle any deviation of the grammar. Furthermore, there are 
some utterances, which can be applied to only specific yes/no-questions like for example "Don't".

For recognizing \emph{days}, I implemented a somewhat more flexible grammar. All possible keywords (in 
this case weekdays) are written out in the grammar. The machine will now check if the utterance includes
the keywords instead of comparing them directly to the grammar. This gives the user more flexibility for
saying things like "On Saturday", "Next Saturday" or "I think Saturday would be nice", but it would
also recognize "Not on Saturday". Assuming that the user won't use negations in the answer, this solution
works ok. 

I implemented the most flexible solution for recognizing the \emph{time}. Here, I used a regular
expression to catch many different possible ways of saying the time (e.g. "3 PM", "02:42", "8",
"10 o'clock"). Using more complex regular expressions, can give the user even more flexible ways
of saying something. But still, regular expressions can't represent all natural language and will
fail at a certain point. This applies especially, when the user is not using grammatically correct 
sentences.

\subsection*{unknown possible intents}
Without looking at the code, the user cannot know, which options he has in the \emph{main menu}.
For this, I tried to implement a help option. After invoking this option by saying e.g. "How can 
you help me?", the machine lists all possible options. Their descriptions are defined also in 
the \emph{menuGrammar} and can be easily switched out by the developer. Still, the user does not 
know excactly, which keywords he needs to use, to invoke a specific option, but he already gets
a general overview of what is possible. Furthermore, this only works well if there are only a few 
options. If there are more, another kind of help action would need to be implemented.

\subsection*{repetitive machine answers}
When conversing with a machine, getting always the same sentence as an answer can be tiring.
It would be more natural, if the machine would also variate with it's utterances. I tried to 
implement this in the clarification requests, where I defined multiple different utterances the 
machine could use. When an utterance by the user wasn't understood, the machine will select one 
of these predefined utterances randomly. In this case of general clarification requests,
this was simple to implement. It would be a lot more complex, if the machine should react to 
specific utterances or words the user said or if the machine should inlcude variable information 
in its utterance.


\subsection*{self correction}
Finally, a problem of this kind of dialogue is that the user has no possibility of self corrections.
When he tries to self correct, the machine won't understand anything or even worse in my 
implementation of parsing the day, it would understand the first uttered day instead of the 
corrected day. Correcting this would require a deeper understanding of the parsed sentence.
In this \emph{FSM}, this problem is not too big, since the machine will utter a clarification 
request in most of the states if it couldn't understand the sentence (except for the day).
This \emph{just} feels unnatural for the user.
\end{document}
